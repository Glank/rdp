\documentclass[11pt]{article}
\usepackage{url}
\usepackage{cite}
\usepackage{amsmath}
\usepackage{amsthm}
\usepackage{graphicx}
\graphicspath{{../../umlet/}}

\newtheorem{definition}{Definition}
\newtheorem{theorem}{Theorem}
\newtheorem{notation}{Notation}

\begin{document}

\title{Computing Solutions to Multifacited Natural Language Questions over Semantic Knowledge Bases}
\author{Ernest Kirstein}
\maketitle

Query languages like SPARQL help expert users answer complex
questions about semantic data knowledge bases. Unfortunately, query languages
can only be used by expert users who require extensive training. 
A natural language interface could help non-expert users access the 
semantic web without expert training. The end goal would be to allow
non-expert users to achieve the same level of sophistication 
as an expert query language user. 

Work by Kaufmann and Bernstein \cite{usability} indicates 
that users have a “clear preference” for even limited natural language interfaces 
when compared to keyword or query language interfaces. 
Other recent works \cite{mapping, freya, galitsky}
have shown that simple natural language questions can be translated based 
on known sentence structure using (among other things) NER, named-entity recognition.

These systems are limited - they can only handle simple, direct questions. 
A recent publication by Sharef et al. \cite{issues} outlines obstacles in developing full 
natural language interfaces for the semantic web. That paper notes a particular 
difficulty with parsing multifaceted questions - questions with multiple 
variable, constraints, or operations. This is the gap which my thesis will 
attempt to bridge.

The foundations of both compiler design and natural language processing
have significant overlap \cite{chomsky, reghizzi}. 
However, in practice, there has not been
much synergy between the two disciplines \cite{aho, anatomy, reghizzi}.
I believe cooperation between these astranged fields is necessary to move forward
in either.

My approach will be to build a "natural language compiler" of sorts.
That is, to approach the problem of converting natural language questions into
SPARQL queries just as one might convert Java source code into byte code.
The problems are, or course, an order of magnitude appart in complexity.
So I will only aim to parse a limited range of natural language questions.
But in that range of questions, I hope to show that multifacited questions
can be handeled on a limited basis.

\bibliography{proposal}{}
\bibliographystyle{plain}
\end{document}
