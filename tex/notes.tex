\documentclass[11pt]{article}
\usepackage{url}
\usepackage{cite}
\usepackage{amsmath}

\begin{document}

\title{Notes}
\author{Ernest Kirstein}
\maketitle

\section*{Recursive Descent Parsing}

The core rules for my parser are built off of Dr. Lewis's notes \cite{lewis}.
A grammar is define as an ordered collection of production rules.
My parser uses context-free grammar rules, which are comprised of a
'head' (the single-symbol left hand side of the production rule), and a 'tail'
(one or more symbols comprising the right hand side of the production rule).

These grammars may be 'compiled' using the four procedures:
factoring, substitution, removing left recursion, and removing useless
rules. Let 'decision list' define an ordered list of production rule
choices which produces a parse tree.
As each of these four procedures produces a weakly equivalent grammar,
there exists a mapping for any decision list in a compiled grammar
back into a same-terminal-producing decision list in the pre-compiled (parent) grammer.
My parser keeps track of these inverse transformation rules as performs
it's compilation procedure so that a compiled grammar's decision list can be easily
converted to the initial grammar's equavalent decision list. 

Take this simple grammar for example:
\begin{align}
S &\rightarrow A B\\
A &\rightarrow a\\
A &\rightarrow S A\\
B &\rightarrow b\\
B &\rightarrow S B
\end{align}
It compiles into the weakly equivalent grammar:
\setcounter{equation}{0}
\begin{align}
Z &\rightarrow \epsilon\\
B &\rightarrow b\\
S &\rightarrow a B S'\\
S' &\rightarrow \epsilon\\
A &\rightarrow a Z\\
B &\rightarrow a B S' B\\
S' &\rightarrow a Z B S'\\
Z &\rightarrow b S' A\\
Z &\rightarrow a B S' B S' A
\end{align}

The progression is simple and goes as follows:\\
Grammar 1:
\setcounter{equation}{0}
\begin{align}
S &\rightarrow A B\\
A &\rightarrow a\\
A &\rightarrow S A\\
B &\rightarrow b\\
B &\rightarrow S B
\end{align}
Grammar 2:
\setcounter{equation}{0}
\begin{align}
A &\rightarrow a\\
A &\rightarrow S A\\
B &\rightarrow b\\
B &\rightarrow S B\\
S &\rightarrow a B\\
S &\rightarrow S A B
\end{align}
Grammar 3:
\setcounter{equation}{0}
\begin{align}
A &\rightarrow a\\
A &\rightarrow S A\\
B &\rightarrow b\\
B &\rightarrow S B\\
S &\rightarrow a B S'\\
S' &\rightarrow A B S'\\
S' &\rightarrow ε
\end{align}
Grammar 4:
\setcounter{equation}{0}
\begin{align}
A &\rightarrow a\\
B &\rightarrow b\\
B &\rightarrow S B\\
S &\rightarrow a B S'\\
S' &\rightarrow A B S'\\
S' &\rightarrow ε\\
A &\rightarrow a B S' A
\end{align}
Grammar 5:
\setcounter{equation}{0}
\begin{align}
Z &\rightarrow ε\\
B &\rightarrow b\\
B &\rightarrow S B\\
S &\rightarrow a B S'\\
S' &\rightarrow A B S'\\
S' &\rightarrow ε\\
Z &\rightarrow B S' A\\
A &\rightarrow a Z
\end{align}
Grammar 6:
\setcounter{equation}{0}
\begin{align}
Z &\rightarrow ε\\
B &\rightarrow b\\
S &\rightarrow a B S'\\
S' &\rightarrow A B S'\\
S' &\rightarrow ε\\
Z &\rightarrow B S' A\\
A &\rightarrow a Z\\
B &\rightarrow a B S' B
\end{align}
Grammar 7:
\setcounter{equation}{0}
\begin{align}
Z &\rightarrow ε\\
B &\rightarrow b\\
S &\rightarrow a B S'\\
S' &\rightarrow ε\\
Z &\rightarrow B S' A\\
A &\rightarrow a Z\\
B &\rightarrow a B S' B\\
S' &\rightarrow a Z B S'
\end{align}
Grammar 8:
\setcounter{equation}{0}
\begin{align}
Z &\rightarrow ε\\
B &\rightarrow b\\
S &\rightarrow a B S'\\
S' &\rightarrow ε\\
A &\rightarrow a Z\\
B &\rightarrow a B S' B\\
S' &\rightarrow a Z B S'\\
Z &\rightarrow b S' A\\
Z &\rightarrow a B S' B S' A
\end{align}
Grammar 1:
\setcounter{equation}{0}
\begin{align}
S &\rightarrow A B\\
A &\rightarrow a\\
A &\rightarrow S A\\
B &\rightarrow b\\
B &\rightarrow S B
\end{align}
Grammar 2:
\setcounter{equation}{0}
\begin{align}
A &\rightarrow a\\
A &\rightarrow S A\\
B &\rightarrow b\\
B &\rightarrow S B\\
S &\rightarrow a B\\
S &\rightarrow S A B
\end{align}
Grammar 3:
\setcounter{equation}{0}
\begin{align}
A &\rightarrow a\\
A &\rightarrow S A\\
B &\rightarrow b\\
B &\rightarrow S B\\
S &\rightarrow a B S'\\
S' &\rightarrow A B S'\\
S' &\rightarrow \epsilon
\end{align}
Grammar 4:
\setcounter{equation}{0}
\begin{align}
A &\rightarrow a\\
B &\rightarrow b\\
B &\rightarrow S B\\
S &\rightarrow a B S'\\
S' &\rightarrow A B S'\\
S' &\rightarrow \epsilon\\
A &\rightarrow a B S' A
\end{align}
Grammar 5:
\setcounter{equation}{0}
\begin{align}
Z &\rightarrow \epsilon\\
B &\rightarrow b\\
B &\rightarrow S B\\
S &\rightarrow a B S'\\
S' &\rightarrow A B S'\\
S' &\rightarrow \epsilon\\
Z &\rightarrow B S' A\\
A &\rightarrow a Z
\end{align}
Grammar 6:
\setcounter{equation}{0}
\begin{align}
Z &\rightarrow \epsilon\\
B &\rightarrow b\\
S &\rightarrow a B S'\\
S' &\rightarrow A B S'\\
S' &\rightarrow \epsilon\\
Z &\rightarrow B S' A\\
A &\rightarrow a Z\\
B &\rightarrow a B S' B
\end{align}
Grammar 7:
\setcounter{equation}{0}
\begin{align}
Z &\rightarrow \epsilon\\
B &\rightarrow b\\
S &\rightarrow a B S'\\
S' &\rightarrow \epsilon\\
Z &\rightarrow B S' A\\
A &\rightarrow a Z\\
B &\rightarrow a B S' B\\
S' &\rightarrow a Z B S'
\end{align}
Grammar 8:
\setcounter{equation}{0}
\begin{align}
Z &\rightarrow \epsilon\\
B &\rightarrow b\\
S &\rightarrow a B S'\\
S' &\rightarrow \epsilon\\
A &\rightarrow a Z\\
B &\rightarrow a B S' B\\
S' &\rightarrow a Z B S'\\
Z &\rightarrow b S' A\\
Z &\rightarrow a B S' B S' A
\end{align}

\bibliography{notes}{}
\bibliographystyle{plain}
\end{document}
