\documentclass[11pt]{article}
\usepackage{url}
\usepackage{cite}
\usepackage{amsmath}
\usepackage{amsthm}
\usepackage{graphicx}
\graphicspath{{../../umlet/}}

\begin{document}

\title{Transforming Grammers and the Effects on Parse Trees}
\author{Ernest Kirstein}
\maketitle

Implementing a parser often requires a grammar with certain properties.
\cite{lewis, compiler}
There are also special properties that a grammer can have which will make
parsing more efficient. \cite{compiler, reghizzi}

But modifying a grammar can be expensive from a software engineering perspective.
Using a non-intuitive form of the grammer will make implementing compilation 
more difficult since the parse trees produced using the modified grammer will
be different than those one would expect from the natural, unmodified form of the
grammar.

An especially expensive engineering problem would be modifying an already-implemented
compiler to use a newly discovered grammar property. Implementing that change
would also require developers to change both the parser and the code generator
which uses the output of the parser.

I propose a radical change - a way to avoid that highly coupled design. Let parsers
use a special modified grammer and let the compiler use the more natural form of
the grammar. Impossible? No. Impractical? Maybe.

\bibliography{transform}{}
\bibliographystyle{plain}
\end{document}
